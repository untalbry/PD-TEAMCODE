\section{Requeriimientos de la primer etapa }

\begin{longtable}{|>{\raggedright\arraybackslash}p{3cm}|>{\raggedright\arraybackslash}p{4cm}|>{\raggedright\arraybackslash}p{7cm}|}
    \hline
    \textbf{ID} & \textbf{Requerimiento} & \textbf{Descripción} \\
    \hline
    \endfirsthead
    \hline
    \textbf{ID} & \textbf{Requerimiento} & \textbf{Descripción} \\
    \hline
    \endhead
    \hline
    \endfoot
    \hline
    \endlastfoot
    REQ-TC-01 & Crear cuenta & Un usuario nuevo podrá crear un perfil nuevo. \\
    \hline
    REQ-TC-02 & Iniciar sesión & Un usuario que ya está en la plataforma podrá ingresar con sus credenciales. \\
    \hline
    REQ-TC-03 & Consultar datos personales del usuario & Un usuario puede consultar sus datos personales, tales como su nombre de usuario, correo electrónico, foto de perfil, etc. \\
    \hline
    REQ-TC-04 & Editar datos de usuario & Un usuario puede cambiar sus datos personales, tales como nombre del usuario, correo electrónico, foto de perfil, etc. \\
    \hline
    REQ-TC-05 & Eliminar cuenta & Un usuario podrá eliminar su cuenta, con la advertencia de que al hacerlo se borrarán permanentemente todos sus datos y progreso del sistema. \\
    \hline
    REQ-TC-06 & Crear un grupo & Permite a los usuarios crear grupos donde se publicarán diariamente ejercicios de LeetCode. \\
    \hline
    REQ-TC-07 & Invitar al grupo & Una vez creado un grupo, los usuarios podrán invitar a otros mediante un enlace de invitación. \\
    \hline
    REQ-TC-08 & Unirse al grupo & Un usuario se puede unir a un grupo a través de un enlace de invitación otorgado por otro usuario. \\
    \hline
    REQ-TC-09 & Obtención de LeetCode por ELO & Un agente obtendrá automáticamente al inicio del día un ejercicio algorítmico diario de acuerdo al ELO calculado del grupo. \\
    \hline
    REQ-TC-10 & Publicar un reto de LeetCode & Publicar diariamente un ejercicio de LeetCode en la plataforma, incluyendo los siguientes atributos: id, nombre, dificultad, descripción, link, fecha de obtención. \\
    \hline
    REQ-TC-11 & Marcar como “logrado” el ejercicio & Cada usuario deberá marcar en el reto publicado en un grupo que logró completar el ejercicio. \\
    \hline
    REQ-TC-12 & Marcar como “no logrado” el ejercicio & Cada usuario deberá marcar el reto publicado en un grupo que no logró completar el ejercicio. \\
    \hline
    REQ-TC-13 & Actualizar ELO de un usuario & Al marcar un ejercicio ya sea como “logrado” o “no logrado” se deberá actualizar el ELO del usuario incrementando su puntaje o disminuyéndolo. \\
    \hline
    REQ-TC-14 & Actualizar ELO del grupo & Cuando todos los miembros de un grupo completen un ejercicio, o al final del día, se deberá actualizar el ELO del grupo en función de sus respuestas. \\
    \hline
    REQ-TC-15 & Consultar desempeño por problema & El sistema debe permitir consultar el desempeño de los usuarios en un problema específico de LeetCode. \\
    \hline
    REQ-TC-16 & Consultar retos de LeetCode por semana & Consultar una lista de ejercicios de LeetCode publicados durante la última semana. \\
    \hline
    REQ-TC-17 & Consultar reto de LeetCode por número de día & Consultar un ejercicio de LeetCode específico, seleccionándolo por el número de día correspondiente. \\
    \hline
    REQ-TC-18 & Consultar desempeño del grupo & Consultar el desempeño general de los usuarios en los problemas resueltos. \\
    \hline
    REQ-TC-19 & Eliminar un usuario del grupo & El administrador del grupo puede eliminar a un usuario específico. \\
    \hline
    REQ-TC-20 & Eliminar usuario por inactividad & Eliminar un usuario después de no iniciar sesión en un periodo de 30 días continuos. Esto para evitar la descompensación del ELO por usuarios inactivos en un grupo. \\
    \hline
    REQ-TC-21 & Asignar rol de administrador a un usuario & Un administrador de grupo puede otorgar el rol de administrador a otro usuario. \\
    \hline
    REQ-TC-22 & Salir de un grupo de LeetCode & Abandonar un grupo de LeetCode al que se hayan unido. \\
\end{longtable}
