\documentclass[10pt, oneside, openany]{book} 
\usepackage[utf8]{inputenc}
\usepackage{graphicx}
\usepackage{xcolor}
\usepackage{lipsum}
\usepackage{titling}
\usepackage{titlesec} 
\usepackage{helvet}
\usepackage{longtable}
\usepackage{booktabs}
\usepackage{array}
\renewcommand{\familydefault}{\sfdefault} 
% Ajustes de márgenes
\usepackage[a4paper, margin=1in]{geometry} 

% Ajustes de espacio entre elementos
\setlength{\abovecaptionskip}{10pt}
\setlength{\belowcaptionskip}{10pt}
\setlength{\textfloatsep}{10pt}

\begin{document}

\title{\textbf{\huge Sistema de control de seguimiento de ejercicios algoritmicos \\[0.5em] \LARGE 1-PDS \\ \Large Propuesta de desarrollo de software}}
\author{Bryan Castro Sánchez}
\date{\today}

    \maketitle
    \vspace{-1em} 
    \begin{center}
        \fbox{\textbf{TEAMCODE--Etapa 1-PDS \ Versión 1.0.0}}
    \end{center}

    \begin{center}
        \textcolor{blue}{\textbf{Binary Brains}} \\[0.3em]
        \textit{Club de desarrollo de software y algoritmia} \\[2em]
        \date{\today}
    \end{center}

    \newpage
    % ----------------------------  Resumen ejecutivo ---------------------------- %
    \chapter{Resumen ejecutivo}
    \label{ch:resumen}
    La presente propuesta es para el desarrollo de una plataforma web que permita a los usuarios practicar algoritmos a través de retos diarios de LeetCode en grupos con funcionalidades específicas de interacción y evaluación de desempeño. El sistema permitirá a los usuarios crear cuentas, unirse a grupos, resolver problemas diarios y obtener una puntuación (ELO) basada en su rendimiento.

El proyecto será desarrollado con Flask en el backend y React en el frontend, lo cual permitirá crear una plataforma escalable y eficiente que fomente la práctica diaria de algoritmos en un entorno colaborativo.

\section{Intención del documento}
    En este documento se refleja el resultado del análisis y diseño del sistema. Tiene como objetivo mostrar a detalle los términos, reglas de negocio, funcionalidades, interfaces y mensajes correspondientes a cada uno de los procesos identificados, los cuales también son descritos a través de un diagrama y una descripción breve.\\
	Va dirigido al personal involucrado en las tareas de dirección, análisis, arquitectura, desarrollo y pruebas del sistema para la realización de su trabajo.
    % ----------------------------  Contexto ---------------------------- %
    \chapter{Antecedentes y contexto}
    \label{ch:contexto}
    La práctica de algoritmos en plataformas como LeetCode es fundamental para el desarrollo de habilidades de resolución de problemas y es muy valorada en el ámbito profesional. 
Sin embargo, no todas las plataformas permiten a los usuarios colaborar o motivarse dentro de un grupo, con un sistema de evaluación que refleje su progreso individual y grupal.
Este sistema busca cubrir esa necesidad, permitiendo que los usuarios interactúen y se motiven a mejorar sus habilidades algorítmicas. Se utilizará una gente de ingeligencia artificial
a con el fin de poder evaluar el nivel del usuario .


    % ----------------------------  Objetivos ---------------------------- %
    \chapter{Objetivos}
    \label{ch:objetivos}
    \begin{itemize}
    \item Crear una plataforma intuitiva que permita a los usuarios practicar algoritmos en grupos de LeetCode.
    \item Implementar un sistema de puntaje ELO que se ajuste según el desempeño del usuario y del grupo.
    \item Permitir la consulta y gestión de usuarios y grupos, incluyendo administración y roles.
    \item Fomentar la práctica constante de algoritmos y motivar la mejora continua.
\end{itemize}

    % ---------------------------- Alcance del proyecto ---------------------------- %
    \chapter{Alcance del Proyecto}
    \label{ch:Alcance}
    \begin{itemize}
    \item \textbf{Funcionalidades del Usuario:} Crear cuenta, iniciar sesión, consultar y editar perfil, eliminar cuenta.
    \item \textbf{Gestión de Grupos:} Crear grupos, invitar y unirse a grupos mediante enlaces, y gestionar la membresía.
    \item \textbf{Interacción con Retos de LeetCode:} Publicación automática diaria de retos según el ELO del grupo, seguimiento del desempeño de los usuarios en los retos.
    \item \textbf{Sistema de Puntaje ELO:} Cálculo y actualización de ELO para usuarios y grupos en función del rendimiento en los ejercicios de LeetCode.
    \item \textbf{Consulta de Desempeño:} Consultar desempeño individual y grupal, además de visualizar retos y resultados específicos.
\end{itemize}

    % ---------------------------- Solución ---------------------------- %
    \chapter{Solucion}
    \label{ch:solución}
    \textbf{Arquitectura de la Aplicación:}
\begin{itemize}
    \item \textbf{Backend (Flask):} Se encargará de gestionar el acceso a la base de datos, el cálculo del ELO, y la lógica de negocios para la gestión de usuarios, grupos y retos de LeetCode.
    \item \textbf{Frontend (React):} Interfaz amigable e intuitiva para que los usuarios puedan interactuar con el sistema, visualizar retos y estadísticas, y gestionar su cuenta y grupos.
\end{itemize}

    % ---------------------------- Requerimientos ---------------------------- %
    \chapter{Requerimientos del sistema}
    \label{ch:requerimientos}
    \section{Requeriimientos de la primer etapa }

\begin{longtable}{|>{\raggedright\arraybackslash}p{3cm}|>{\raggedright\arraybackslash}p{4cm}|>{\raggedright\arraybackslash}p{7cm}|}
    \hline
    \textbf{ID} & \textbf{Requerimiento} & \textbf{Descripción} \\
    \hline
    \endfirsthead
    \hline
    \textbf{ID} & \textbf{Requerimiento} & \textbf{Descripción} \\
    \hline
    \endhead
    \hline
    \endfoot
    \hline
    \endlastfoot
    REQ-TC-01 & Crear cuenta & Un usuario nuevo podrá crear un perfil nuevo. \\
    \hline
    REQ-TC-02 & Iniciar sesión & Un usuario que ya está en la plataforma podrá ingresar con sus credenciales. \\
    \hline
    REQ-TC-03 & Consultar datos personales del usuario & Un usuario puede consultar sus datos personales, tales como su nombre de usuario, correo electrónico, foto de perfil, etc. \\
    \hline
    REQ-TC-04 & Editar datos de usuario & Un usuario puede cambiar sus datos personales, tales como nombre del usuario, correo electrónico, foto de perfil, etc. \\
    \hline
    REQ-TC-05 & Eliminar cuenta & Un usuario podrá eliminar su cuenta, con la advertencia de que al hacerlo se borrarán permanentemente todos sus datos y progreso del sistema. \\
    \hline
    REQ-TC-06 & Crear un grupo & Permite a los usuarios crear grupos donde se publicarán diariamente ejercicios de LeetCode. \\
    \hline
    REQ-TC-07 & Invitar al grupo & Una vez creado un grupo, los usuarios podrán invitar a otros mediante un enlace de invitación. \\
    \hline
    REQ-TC-08 & Unirse al grupo & Un usuario se puede unir a un grupo a través de un enlace de invitación otorgado por otro usuario. \\
    \hline
    REQ-TC-09 & Obtención de LeetCode por ELO & Un agente obtendrá automáticamente al inicio del día un ejercicio algorítmico diario de acuerdo al ELO calculado del grupo. \\
    \hline
    REQ-TC-10 & Publicar un reto de LeetCode & Publicar diariamente un ejercicio de LeetCode en la plataforma, incluyendo los siguientes atributos: id, nombre, dificultad, descripción, link, fecha de obtención. \\
    \hline
    REQ-TC-11 & Marcar como “logrado” el ejercicio & Cada usuario deberá marcar en el reto publicado en un grupo que logró completar el ejercicio. \\
    \hline
    REQ-TC-12 & Marcar como “no logrado” el ejercicio & Cada usuario deberá marcar el reto publicado en un grupo que no logró completar el ejercicio. \\
    \hline
    REQ-TC-13 & Actualizar ELO de un usuario & Al marcar un ejercicio ya sea como “logrado” o “no logrado” se deberá actualizar el ELO del usuario incrementando su puntaje o disminuyéndolo. \\
    \hline
    REQ-TC-14 & Actualizar ELO del grupo & Cuando todos los miembros de un grupo completen un ejercicio, o al final del día, se deberá actualizar el ELO del grupo en función de sus respuestas. \\
    \hline
    REQ-TC-15 & Consultar desempeño por problema & El sistema debe permitir consultar el desempeño de los usuarios en un problema específico de LeetCode. \\
    \hline
    REQ-TC-16 & Consultar retos de LeetCode por semana & Consultar una lista de ejercicios de LeetCode publicados durante la última semana. \\
    \hline
    REQ-TC-17 & Consultar reto de LeetCode por número de día & Consultar un ejercicio de LeetCode específico, seleccionándolo por el número de día correspondiente. \\
    \hline
    REQ-TC-18 & Consultar desempeño del grupo & Consultar el desempeño general de los usuarios en los problemas resueltos. \\
    \hline
    REQ-TC-19 & Eliminar un usuario del grupo & El administrador del grupo puede eliminar a un usuario específico. \\
    \hline
    REQ-TC-20 & Eliminar usuario por inactividad & Eliminar un usuario después de no iniciar sesión en un periodo de 30 días continuos. Esto para evitar la descompensación del ELO por usuarios inactivos en un grupo. \\
    \hline
    REQ-TC-21 & Asignar rol de administrador a un usuario & Un administrador de grupo puede otorgar el rol de administrador a otro usuario. \\
    \hline
    REQ-TC-22 & Salir de un grupo de LeetCode & Abandonar un grupo de LeetCode al que se hayan unido. \\
\end{longtable}

\end{document}
