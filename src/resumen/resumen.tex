La presente propuesta es para el desarrollo de una plataforma web que permita a los usuarios practicar algoritmos a través de retos diarios de LeetCode en grupos con funcionalidades específicas de interacción y evaluación de desempeño. El sistema permitirá a los usuarios crear cuentas, unirse a grupos, resolver problemas diarios y obtener una puntuación (ELO) basada en su rendimiento.

El proyecto será desarrollado con Flask en el backend y React en el frontend, lo cual permitirá crear una plataforma escalable y eficiente que fomente la práctica diaria de algoritmos en un entorno colaborativo.

\section{Intención del documento}
    En este documento se refleja el resultado del análisis y diseño del sistema. Tiene como objetivo mostrar a detalle los términos, reglas de negocio, funcionalidades, interfaces y mensajes correspondientes a cada uno de los procesos identificados, los cuales también son descritos a través de un diagrama y una descripción breve.\\
	Va dirigido al personal involucrado en las tareas de dirección, análisis, arquitectura, desarrollo y pruebas del sistema para la realización de su trabajo.